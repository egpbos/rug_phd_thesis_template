%!TEX root = thesis.tex
%%%%%%%%%%%%%%%%%%%%%%%%%%%%%%%%%%%%%%%%%%%%%%%%%%%%%%%%%%%%%%%%%%%%%%%%
% Kapteyn Institute Thesis Model, revised by Bruno Letarte 2006
% Later used by Matias Arrigoni (2010), Thomas Martinsson (2011), Stephan Peters
% (2014) and Patrick Bos (2016).
%
% make PDF directly with pdflatex, dvi can mess up the page size.
% pdflatex -output-format=pdf
% for this, make all figures as pdf (or jpeg, png) with: epstopdf.
% UPDATE (P.B.): don't do it manually, just use the epstopdf package!
% Doesn't seem to work though... oh well.
%
% dutch thesis size: 170mm x 240mm
%
%%%%%%%%%%%%%%%%%%%%%%%%%%%%%%%%%%%%%%%%%%%%%%%%%%%%%%%%%%%%%%%%%%%%%%%%
% EGP: No idea what this does, but it was in thesis.tex and I think it should
%      be here.
\DeclareTextCommandDefault{\nobreakspace}{\leavevmode\nobreak\ } 
%%%%%%%%%%%%%%%%%%%%%%%%%%%%%%%%%%%%%%%%%%%%%%%%%%%%%%%%%%%%%%%%%%%%%%%%
% Use {lmodern} instead of {ae,aecompl} if possible.
% Do not use [T1]{fontenc} without one of the two other.
% This makes sure you have scalable fonts with support for special 
% characters, accents, and support for hyphenation in most languages.
%%%%%%%%%%%%%%%%%%%%%%%%%%%%%%%%%%%%%%%%%%%%%%%%%%%%%%%%%%%%%%%%%%%%%%%%
%\usepackage{ae,aecompl} \usepackage[T1]{fontenc}
% EGP: lmodern is no longer being updated, but also it doesn't support
%      smallcaps in section headers. cm-super does and it is automatically
%      used when [T1]{fontenc} is loaded (no need to do a usepackage).
% \usepackage{lmodern}
\usepackage[T1]{fontenc}

% EGP: following should fix small caps in titles (but doesn't) (from
% http://tex.stackexchange.com/a/22241/66614):
% \normalfont % needed to load cmss
% \DeclareFontShape{T1}{cmss}{bx}{sc} { <-> ssub * cmr/bx/sc }{}

%%%%%%%%%%%%%%%%%%%%%%%%%%%%%%%%%%%%%%%%%%%%%%%%%%%%%%%%%%%%%%%%%%%%%%%%
% with TeXLive 2005, pdfeTeX replaces TeX and creates both .dvi and .pdf
% use only if you use pdfeTeX as your latex compiler (check bin link)
%%%%%%%%%%%%%%%%%%%%%%%%%%%%%%%%%%%%%%%%%%%%%%%%%%%%%%%%%%%%%%%%%%%%%%%%
%\pdfadjustspacing=1    %use same spacing as old LaTeX
\usepackage{pdflscape}  %use lscape if not using pdfeTeX for landscape

%%%%%%%%%%%%%%%%%%%%%%%%%%%%%%%%%%%%%%%%%%%%%%%%%%%%%%%%%%%%%%%%%%%%%%%%
% Other packages and commands, optional
%%%%%%%%%%%%%%%%%%%%%%%%%%%%%%%%%%%%%%%%%%%%%%%%%%%%%%%%%%%%%%%%%%%%%%%%
\usepackage{graphicx}  %to include figures
\usepackage{epstopdf}  % must be after graphicx, otherwise may not work
\usepackage[space]{grffile}   % for figure filenames with spaces
\usepackage{longtable} %tables on more than one page
\usepackage{tabularx}  %tables with auto-breaking lines, etc.
\usepackage{booktabs}  %prettier tables
\usepackage{rotating}  %rotate tables
\usepackage{sidecap}   %use with \begin{SCfigure}
\usepackage{lineno}    %use with \linenumbers* to start lines at 1.
\usepackage{lettrine}  %use with \lettrine{F}{irst} word of a chapter
\usepackage{soul}      %use for \st{text} (strike-out)
\usepackage{subcaption}    % for subfloats
\usepackage{color}
\usepackage{amsmath, amssymb} %for math symbols
% \usepackage{stmaryrd}     %for math symbols; EGP: removed, gave me an error, wasn't using it anyway
\usepackage{bm}       % for bold math symbols, http://tex.stackexchange.com/questions/3238/bm-package-versus-boldsymbol
% EGP: I was using boldsymbol for vectors before, but this gave some issues in
%      the list of figures, somehow (warnings, not errors, but still annoying).
\usepackage[perpage,symbol*,hang]{footmisc} %better footnotes
% \usepackage[hang]{footmisc} %better footnotes % EGP: double
\usepackage[title,titletoc]{appendix}       %\begin{subappendices}
%\usepackage{appendix}
\def\labelitemi{\ensuremath\bullet}  % to avoid a font warning
\def\labelitemii{\ensuremath\bullet} % to avoid a font warning

% EGP: (voor barcode_rsd chapter):
\usepackage{amsfonts}
\usepackage{mathrsfs} % mathscr commando
\usepackage[usenames,dvipsnames]{xcolor} % voor gekleurde tekst en symbolen in vergelijkingen
\usepackage{mathtools} % \mathclap and other things to make equations prettier
\usepackage{empheq} % emphasize equations, logisch

% EGP: gothic fonts (\textfrak, \textswab, \textgoth):
\usepackage{yfonts}

% EGP: make toc compacter
\usepackage{tocloft}

%%%%%%%%%%%%%%%%%%%%%%%%%%%%%%%%%%%%%%%%%%%%%%%%%%%%%%%%%%%%%%%%%%%%%%%%
% use only if you have a recent version of caption: >= 2004/07/16 v3.0c
% ex: \captionsetup[longtable]{font=small, justification=centering}
%%%%%%%%%%%%%%%%%%%%%%%%%%%%%%%%%%%%%%%%%%%%%%%%%%%%%%%%%%%%%%%%%%%%%%%%

\definecolor{RUGred}{RGB}{0,0,0}
\definecolor{RUGredgrey}{RGB}{100,100,100}
\definecolor{RUGblue}{RGB}{0,0,0}
\definecolor{RUGbluegrey}{RGB}{100,100,100}

%%%%%%%%%%%%%%%%%%%%%%%%%%%%%%%%%%%%%%%%%%%%%%%%%%%%%%%%%%%%%%%%%%%%%%%%
% Fancy style with bar on top, no page number on bottom
%%%%%%%%%%%%%%%%%%%%%%%%%%%%%%%%%%%%%%%%%%%%%%%%%%%%%%%%%%%%%%%%%%%%%%%%
\usepackage{fancyhdr}
\pagestyle{fancyplain} 
\cfoot{} % Remove page number at the bottom of the page
%\setlength{\headwidth}{\textwidth} %reset this after geometry package
\renewcommand{\chaptermark}[1]{\markboth{chapter \thechapter:~ #1}{}}
\renewcommand{\sectionmark}[1]{\markright{\thesection:~ #1}}
\lhead[\fancyplain{}{\thepage}]{\fancyplain{}{\textsc\rightmark}}
\rhead[\fancyplain{}{\textsc\leftmark}]{\fancyplain{}{\thepage}}



\pagestyle{fancy}
\fancyhead[LE,RO]{\color{RUGredgrey}\slshape\thepage}
\fancyhead[RE]{\color{RUGredgrey}\slshape \leftmark}
\fancyhead[LO]{\color{RUGredgrey}\slshape \rightmark}




\usepackage{titlesec}
\usepackage[percent]{overpic}




\titleformat{\section}{\color{RUGred}\normalfont\Large\bfseries}{\color{RUGred}\thesection}{1em}{}
\titleformat{\subsection}{\color{RUGred}\normalfont\bfseries}{\color{RUGred}\thesubsection}{1em}{}
\titleformat{\subsubsection}{\color{RUGred}\normalfont\bfseries}{}{1em}{}
\titlespacing{\section}{0pt}{10pt}{3pt}
\titlespacing{\subsection}{0pt}{10pt}{0pt}
\titlespacing{\subsubsection}{0pt}{10pt}{0pt}
\setlength{\parindent}{0cm}
\newcommand{\hsp}{\hspace{10pt}}
%\titleformat{\chapter}[hang]{\Huge\bfseries}{\thechapter\hsp\textcolor{RUGred}{|}\hsp}{0pt}{\Huge\bfseries}




%\titleformat{\chapter}[hang]{\Huge\bfseries}{\thechapter \hsp \resizebox{!}{22pt}{\includegraphics{ch_frontandback/thickslash.pdf}} \hsp}{0pt}{\Huge\bfseries}
%\titleformat{\chapter}[hang]{\Huge\bfseries}{\resizebox{0.3\textwidth}{!}{\includegraphics{ch_frontandback/thickslash.pdf}}\thechapter}{0pt}{\Huge\bfseries}

%
%\titleformat{\chapter}[hang]{\Huge\bfseries}{\begin{overpic}[scale=.35,tics=10]{ch_frontandback/thickslash.pdf}\put(10,50){\huge \bfseries \thechapter}\end{overpic}}{0pt}{\Huge\bfseries}


\titlespacing{\chapter}{}{10pt}{}[23pt]





% \renewcommand{\headheight}{13.6pt}
%%%%%%%%%%%%%%%%%%%%%%%%%%%%%%%%%%%%%%%%%%%%%%%%%%%%%%%%%%%%%%%%%%%%%%%%
% Language and Bibliography style
%%%%%%%%%%%%%%%%%%%%%%%%%%%%%%%%%%%%%%%%%%%%%%%%%%%%%%%%%%%%%%%%%%%%%%%%
\usepackage[dutch,english]{babel}
\usepackage[]{natbib}

\bibpunct{(}{)}{;}{a}{}{,}

%%%%%%%%%%%%%%%%%%%%%%%%%%%%%%%%%%%%%%%%%%%%%%%%%%%%%%%%%%%%%%%%%%%%%%%%
% this hyperref packages must be loaded at the end.
%%%%%%%%%%%%%%%%%%%%%%%%%%%%%%%%%%%%%%%%%%%%%%%%%%%%%%%%%%%%%%%%%%%%%%%%
%\ifpdf
%%%... do things, if pdfTeX is running in pdf mode ...
%\usepackage[pdftex,bookmarksopen,bookmarksnumbered]{hyperref}
%%%\pdfminorversion 6 % if need pdf 1.6 instead of default 1.4
%\else
%%%... other TeX or pdfTeX in dvi mode ...
%\usepackage[dvips, bookmarksopen,bookmarksnumbered]{hyperref}
%\fi
% \usepackage[bookmarksopen,bookmarksnumbered]{hyperref}
\usepackage[bookmarksopen,
            bookmarksnumbered,
            colorlinks,
            linkcolor=blue,
            citecolor=blue,
            urlcolor=blue,
            pdfauthor={Bos et al.},
            linktocpage=true,
            draft % set this to remove hyperlinks for the print version
            ]{hyperref} % klikbare referenties, citaties en inhoudsopgave, van http://arxiv.org/abs/1405.4285

\hypersetup{pdfauthor  = {E. G. Patrick Bos},
            pdftitle   = {Clusters, Voids and Reconstructions of the Cosmic Web},
            pdfsubject = {PhD Thesis, Kapteyn Astronomical Institute, University of Groningen, 2016},
            pdfkeywords= {cosmology, large scale structure, dark energy, voids, Bayesian, reconstructions}}

%%%%%%%%%%%%%%%%%%%%%%%%%%%%%%%%%%%%%%%%%%%%%%%%%%%%%%%%%%%%%%%%%%%%%%%%
% Geometry must be loaded after hyperref for proper papersize in pdf
% redefine the \setlength{\headwidth}{\textwidth} for new textwidth!
%%%%%%%%%%%%%%%%%%%%%%%%%%%%%%%%%%%%%%%%%%%%%%%%%%%%%%%%%%%%%%%%%%%%%%%%
\usepackage[dvips=false,pdftex=false,vtex=false]{geometry} %remove default
\geometry{twoside, %needed for outer+inner margins
          %showframe %remove for final print
          nofoot, %no bottom pages
          nomarginpar,
          paperwidth=174mm,  % include 4mm for bleed
          paperheight=244mm, % include 4mm for bleed
          layoutwidth=170mm,
          layoutheight=240mm,
          layouthoffset=2mm, % bleed; printer will cut this off; used for edge-images
          layoutvoffset=2mm, % bleed
          headsep=5mm, %between top bar and text
          %headheight=5mm, %no need to set it, will expand normally
          top=25mm, %includes headsep + headheight
          bottom=15mm,
          inner=22mm,
          outer=18mm,
          portrait}


\setlength{\headwidth}{\textwidth} %header line on top of fancyheader
\setlength{\headheight}{13.6pt} %header line on top of fancyheader

%%%%%%%%%%%%%%%%%%%%%%%%%%%%%%%%%%%%%%%%%%%%%%%%%%%%%%%%%%%%%%%%%%%%%%%%
% Make a frame around papersize, important to load after {geometry}
% !!! REMOVE all of that for final version !!!
% to test printing a real page size, use a crop in this way:
%\usepackage[pdftex,frame,noinfo]{crop} %no a4, no info, no center
% or (with dvi):
%\usepackage[dvips, frame,noinfo]{crop} %no a4, no info, no center
% dvips -Ppdf -G0 -T 170mm,240mm thesis.dvi && ps2pdf thesis.ps
%%%%%%%%%%%%%%%%%%%%%%%%%%%%%%%%%%%%%%%%%%%%%%%%%%%%%%%%%%%%%%%%%%%%%%%%

% \ifpdf %output-format=pdf
% \usepackage[pdftex,a4,center,frame]{crop} %center the small page on a A4paper:
\usepackage[pdftex,frame,noinfo]{crop} %no a4, no info, no center
% \else %output-format=dvi: 
 % \usepackage[pdftex,a4,center,noinfo]{crop} %center the small page on a A4paper:
%\usepackage[dvips,a4,center,frame,noinfo]{crop} %no a4, no info, no center
 %dvips -Ppdf -G0 -t A4size thesis.dvi && ps2pdf -sPAPERSIZE=a4 thesis.ps
% \fi



\usepackage{setspace}


%%%%%%%%%%%%%%%%%%%%%%%%%%%%%%%%%%%%%%%%%%%%%%%%%%%%%%%%%%%%%%%%%%%%%%%%
% Defines macros for personal use
%%%%%%%%%%%%%%%%%%%%%%%%%%%%%%%%%%%%%%%%%%%%%%%%%%%%%%%%%%%%%%%%%%%%%%%%
\newcommand{\fei}{Fe\,{\sc i}}
\newcommand{\feii}{Fe\,{\sc ii}}
\newcommand{\tii}{Ti\,{\sc i}}
\newcommand{\tiii}{Ti\,{\sc ii}}
\newcommand{\gf}{\textit{\sffamily{g}\rmfamily{f}}}
\newcommand{\logg}{\ensuremath{\log g}}
\newcommand{\msol}{\ensuremath{\mathcal{M}_{\odot}}}
\newcommand{\msun}{\mbox{${\rm M}_{\odot}$}}

% -----------------------------------------------------------------------------------------------
% Nomenclature used in the Disk-Mass series

% Latest update from TPKM nov10

% Shorthands
\newcommand{\sect}{\S~}
\newcommand{\sects}{\S\S~}
\newcommand{\dpr}{^{\prime\prime}}
\newcommand{\kms}{{km s$^{-1}$}}
\newcommand{\ang}{{\AA}}
\newcommand{\ston}{S/N}
\newcommand{\mston}{\langle S/N\rangle}
\newcommand{\llsol}{\mathcal{L}_\odot^{\lambda}}
\newcommand{\lksol}{\mathcal{L}_\odot^K}
\newcommand{\sdu}{$\msol$ pc$^{-2}$}
\newcommand{\denu}{$\msol$ kpc$^{-3}$}
\newcommand{\muu}{mag arcsec$^{-2}$}

% Ions
\newcommand{\halp}{H$\alpha$}
\newcommand{\hbet}{H$\beta$}

\usepackage{relsize}

%\newcommand\ion[2]{#1$\;${\scshape{#2}}}%

%\newcommand{\hi}{\textrm{H$_\textrm{I}$}}
%\newcommand{\hi}{\textrm{\textsc{H{\smaller}I}}}

% Define HI
%\ifmmode \newcommand{\hi}{\textrm{\textsc{HI}} }%
%\else \newcommand{\hi}{{\textsc{H{\smaller}I}} }%
%\fi

\newcommand{\hi}{%
  \relax
  \ifmmode
    \textrm{\textsc{HI}}
  \else
    \textsc{H{\smaller}I}
  \fi
}





%\newcommand{\hi}{\textsc{Hi} }


\newcommand{\oh}{OH }
\newcommand{\co}{CO }
\newcommand{\oiii}{[O{\scshape iii}]}
\newcommand{\hone}{H{\scshape i}}
\newcommand{\none}{N{\scshape i}}
\newcommand{\nii}{[N{\scshape ii}]}
\newcommand{\sii}{[S{\scshape ii}]}
\newcommand{\mgi}{Mg{\scshape i}}
%\newcommand{\fei}{Fe{\scshape i}}

% Coordinates
\newcommand{\gaz}{\theta}
\newcommand{\pa}{\phi_0}
\newcommand{\saz}{\phi}
\newcommand{\rhi}{R_{\mbox{\small \ion{H}{1}}}}

% Velocities
\newcommand{\vhel}{V_{\rm hel}}
\newcommand{\vsys}{V_{\rm sys}}
\newcommand{\vobs}{V_{\rm obs}}
\newcommand{\vlos}{V_{\rm LOS}}
\newcommand{\vglos}{V_{\rm g}^{\rm LOS}}
\newcommand{\vslos}{V_{\ast}^{\rm LOS}}
\newcommand{\vt}{V_\theta}
\newcommand{\vrad}{V_R}
\newcommand{\vz}{V_z}
\newcommand{\vrot}{V_{\rm rot}}
\newcommand{\vc}{V_{\rm c}}
\newcommand{\vdisk}{V_{\rm disk}}
\newcommand{\vstar}{V_{\ast}}
\newcommand{\vdiskgas}{V^{\rm disk}_{\rm gas}}
\newcommand{\vdiskstar}{V^{\rm disk}_\ast}
\newcommand{\vdiskstarmax}{V^{\rm disk}_{\ast,\rm max}}
\newcommand{\vbulge}{V_{\rm bulge}}
\newcommand{\vhalo}{V_{\rm halo}}
\newcommand{\vdm}{V_{\rm DM}}
\newcommand{\vism}{V_{\rm ISM}}
\newcommand{\vgas}{V_{\rm gas}}
\newcommand{\vsphere}{V_{\rm sphere}}
\newcommand{\vbar}{V_{\rm bar}}


% Velocity Dispersions
\newcommand{\sobs}{\sigma_{\rm obs}}
\newcommand{\slos}{\sigma_{\rm LOS}}
\newcommand{\smaj}{\sigma_{\rm maj}}
\newcommand{\smin}{\sigma_{\rm min}}
\newcommand{\sglos}{\sigma_{\rm g}^{\rm LOS}}
\newcommand{\sslos}{\sigma_{\ast}^{\rm LOS}}
\newcommand{\sinst}{\sigma_{\rm inst}}
\newcommand{\stinst}{\sigma^{\rm inst}_T}
\newcommand{\sginst}{\sigma^{\rm inst}_G}
\newcommand{\sbs}{\sigma_{\rm beam}}
\newcommand{\stm}{\sigma_{\rm tpl}}

% Stellar Velocity Ellipsoid
\newcommand{\sigr}{\sigma_R}
\newcommand{\sigp}{\sigma_{\gaz}}
\newcommand{\sigz}{\sigma_z}
\newcommand{\lrz}{\lambda_{Rz}}

% Photometrics

% Surface Densities
\newcommand{\sddisk}{\Sigma_{\rm dyn}}
\newcommand{\sds}{\Sigma_{\rm \ast}}
\newcommand{\sdhi}{\Sigma_{\mbox{\rm \footnotesize H{\scshape i}}}}
\newcommand{\sdi}{\Sigma_{\rm ion}}
\newcommand{\sda}{\Sigma_{\rm atom}}
\newcommand{\sdm}{\Sigma_{\rm mol}}
\newcommand{\sdmh}{\Sigma_{\rm H_2}}
\newcommand{\sdg}{\Sigma_{\rm gas}}
\newcommand{\sdd}{\Sigma_{\rm dust}}
\newcommand{\sddm}{\Sigma_{\rm DM}}

\newcommand{\rhodisk}{\rho_{\ast}^{\rm disk}}
\newcommand{\rhodm}{\rho_{\rm DM}}
\newcommand{\rhob}{\rho_{\rm b}}

% Masses
\newcommand{\mtotdisk}{\mathcal{M}^{\rm tot}_{\rm disk}}
\newcommand{\mtot}{\mathcal{M}^{\rm tot}_{\rm dyn}}
\newcommand{\mdisk}{\mathcal{M}^{\rm disk}_{\rm dyn}}
\newcommand{\mdiskstar}{\mathcal{M}^{\rm disk}_{\rm \ast}}
\newcommand{\mbulgestar}{\mathcal{M}^{\rm bulge}_{\ast}}
\newcommand{\mhalo}{\mathcal{M}^{\rm halo}_{\rm dyn}}
\newcommand{\mhalodm}{\mathcal{M}^{\rm halo}_{\rm DM}}
\newcommand{\mhalobar}{\mathcal{M}^{\rm halo}_{\rm bar}}
\newcommand{\mhi}{\mathcal{M}_{\mbox{\small \ion{H}{1}}}}
\newcommand{\matom}{\mathcal{M}_{\rm atom}}
\newcommand{\mmol}{\mathcal{M}_{\rm mol}}
\newcommand{\mg}{\mathcal{M}_{\rm gas}}
\newcommand{\mdust}{\mathcal{M}_{\rm dust}}
\newcommand{\mbar}{\mathcal{M}_{\rm bar}}
\newcommand{\mb}{\mathcal{M}_{\rm b}}
\newcommand{\mdm}{\mathcal{M}_{\rm DM}}

% Mass-to-light Ratios
\newcommand{\ml}{\Upsilon}
\newcommand{\mldyn}{\Upsilon_{\rm dyn}}
\newcommand{\mldynl}{\Upsilon_{\rm dyn,\lambda}}
\newcommand{\mldynk}{\Upsilon_{{\rm dyn},K}}
\newcommand{\mldyndisk}{\Upsilon_{\rm dyn}^{\rm disk}}
\newcommand{\mldynldisk}{\Upsilon_{\rm dyn,\lambda}^{\rm disk}}
\newcommand{\mldynkdisk}{\Upsilon_{{\rm dyn},K}^{\rm disk}}
\newcommand{\mls}{\Upsilon_\ast}
\newcommand{\mlsdisk}{\Upsilon_\ast^{\rm disk}}
\newcommand{\mlsl}{\Upsilon_{\ast,\lambda}}
\newcommand{\mlsldisk}{\Upsilon_{\ast,\lambda}^{\rm disk}}
\newcommand{\mlsk}{\Upsilon_{\ast,K}}
\newcommand{\mlsksps}{\Upsilon_{\ast,K}^{\rm SPS}}
\newcommand{\mlskdisk}{\Upsilon_{\ast,K}^{\rm disk}}
\newcommand{\mlsld}{\Upsilon_{\ast,\lambda}^{\rm disk}}
\newcommand{\mlslbl}{\Upsilon_{\ast,\lambda}^{\rm bulge}}

% Disk Maximalities or Mass Fractions
\newcommand{\fdmhalo}{f_{\rm DM}^{\rm halo}}
\newcommand{\fsd}{f_\ast^{\rm disk}}
\newcommand{\Fbulge}{\mathcal{F}_{\ast}^{\rm bulge}}
\newcommand{\Fdisk}{\mathcal{F}_{\ast}^{\rm disk}}
\newcommand{\Fdiskmax}{\mathcal{F}_{\ast,{\rm max}}^{\rm disk}}
\newcommand{\rdmax}{R_{\ast,{\rm max}}^{\rm disk}}
\newcommand{\Fb}{\mathcal{F}_{\rm b}}
\newcommand{\Fbmax}{\mathcal{F}^{\rm b}_{\rm max}}



% Included by TPKM:
\newcommand{\arot}{V_{\rm arot}}
\newcommand{\hr}{h_{R}}
\newcommand{\hz}{h_{z}}
%\newcommand{\hv}{h_{v}}
\newcommand{\hs}{h_{\sigma}}
\newcommand{\vmax}{V_{\rm max}}
\newcommand{\rHI}{R_{\rm HI}}
\newcommand{\mstar}{\mathcal{M}_{\ast}}
\newcommand{\Fbary}{\mathcal{F}_{\rm bary}}
\newcommand{\vbary}{V_{\rm bary}}


\newcommand\arcdeg{\mbox{$^\circ$}}%
\newcommand\arcmin{\mbox{$^\prime$}}%
\newcommand\arcsec{\mbox{$^{\prime\prime}$}}%
\newcommand\fd{\mbox{$.\!\!^{\mathrm d}$}}%
\newcommand\fh{\mbox{$.\!\!^{\mathrm h}$}}%
\newcommand\fm{\mbox{$.\!\!^{\mathrm m}$}}%
\newcommand\fs{\mbox{$.\!\!^{\mathrm s}$}}%
\newcommand\fdg{\mbox{$.\!\!^\circ$}}%
%\newcommand\farcm@mss{\mbox{$.\mkern-4mu^\prime$}}
% \let\farcm\farcm@mss
%\newcommand\farcs@mss{\mbox{$.\!\!^{\prime\prime}$}}
% \let\farcs\farcs@mss


% EGP:
\def\hmpc{~h^{-1} {\rm Mpc}}



% -----------------------------------------------------------------------------------------------

\DeclareMathOperator\erf{erf}



%%%%%%%%%%%%%%%%%%%%%%%%%%%%%%%%%%%%%%%%%%%%%%%%%%%%%%%%%%%%%%%%%%%%%%%%
% Appendix as chapter sections for thesis without using \appendix
% Will be in ToC, for example: numbered 5.A and figs will be 5.A1
%%%%%%%%%%%%%%%%%%%%%%%%%%%%%%%%%%%%%%%%%%%%%%%%%%%%%%%%%%%%%%%%%%%%%%%%
%\begin{subappendices}
%\renewcommand{\thetable}{\arabic{chapter}.\Alph{section}\arabic{table}}
%\setcounter{table}{0}
%\renewcommand{\thefigure}{\arabic{chapter}.\Alph{section}\arabic{figure}}
%\setcounter{figure}{0}
%\section{Extra tables and figures}
%\end{subappendices}

% EGP: with this defined, you don't need to do the above anymore.
%      From http://tex.stackexchange.com/a/120723/66614 
%   Unfortunately, the counterwithin commands break figure numbering in regular
%   sections, so left it out in the end.
% \usepackage{chngcntr}
\usepackage{etoolbox}
\AtBeginEnvironment{subappendices}{% AtBeginEnvironment is from etoolbox
\chapter*{Appendix}
% % \addcontentsline{toc}{chapter}{Appendices}
% \counterwithin{figure}{section}  % from chngcntr
% \counterwithin{table}{section}  % from chngcntr
}
% Can then do just:
% \begin{subappendices}
% \section{Some title for an appendix}
% blablbala
% \section{Some title for an appendix}
% \end{subappendices}


%%%%%%%%%%%%%%%%%%%%%%%%%%%%%%%%%%%%%%%%%%%%%%%%%%%%%%%%%%%%%%%%%%%%%%%%
% Alter some LaTeX defaults for better treatment of figures:
% See p.105 of "TeX Unbound" for suggested values.
% See pp. 199-200 of Lamport's "LaTeX" book for details.
% This is not necessary but does make a better job of placing floats.
%%%%%%%%%%%%%%%%%%%%%%%%%%%%%%%%%%%%%%%%%%%%%%%%%%%%%%%%%%%%%%%%%%%%%%%%

%   General parameters, for ALL pages:
\renewcommand{\topfraction}{0.9}    % max fraction of floats at top
\renewcommand{\bottomfraction}{0.8} % max fraction of floats at bottom

%   Parameters for TEXT pages (not float pages):
\setcounter{topnumber}{2}
\setcounter{bottomnumber}{2}
\setcounter{totalnumber}{4}         % 2 may work better
\setcounter{dbltopnumber}{2}        % for 2-column pages
\renewcommand{\dbltopfraction}{0.9} % fit big float above 2-col. text
\renewcommand{\textfraction}{0.07}  % allow minimal text w. figs

%   Parameters for FLOAT pages (not text pages):
\renewcommand{\floatpagefraction}{0.7}      % require fuller float pages

%   N.B.: floatpagefraction MUST be less than topfraction !!
\renewcommand{\dblfloatpagefraction}{0.7}   % require fuller float pages



\newcommand*\cleartoleftpage{%
  \clearpage
  \ifodd\value{page}\hbox{}\newpage\fi
}




%%%%%%%%%%%%%%%%%%%%%%%%%%%%%%%%%%%%%%%%%%%%%%%%%%%%%%%%%%%%%%%%%%%%%%%%
% Bibliography and bibfile (need aa.bst in directory)
% direct copy from aa.cls for bibtex definitions
% If you don't use bibtex, you don't need that.
%%%%%%%%%%%%%%%%%%%%%%%%%%%%%%%%%%%%%%%%%%%%%%%%%%%%%%%%%%%%%%%%%%%%%%%%
\def\aj{AJ}%
          % Astronomical Journal
\def\actaa{Acta Astron.}%
          % Acta Astronomica
\def\araa{ARA\&A}%
          % Annual Review of Astron and Astrophys
\def\apj{ApJ}%
          % Astrophysical Journal
\def\apjl{ApJ}%
          % Astrophysical Journal, Letters
\def\apjs{ApJS}%
          % Astrophysical Journal, Supplement
\def\ao{Appl.~Opt.}%
          % Applied Optics
\def\apss{Ap\&SS}%
          % Astrophysics and Space Science
\def\aap{A\&A}%
          % Astronomy and Astrophysics
\def\aapr{A\&A~Rev.}%
          % Astronomy and Astrophysics Reviews
\def\aaps{A\&AS}%
          % Astronomy and Astrophysics, Supplement
\def\azh{AZh}%
          % Astronomicheskii Zhurnal
\def\baas{BAAS}%
          % Bulletin of the AAS
\def\bac{Bull. astr. Inst. Czechosl.}%
          % Bulletin of the Astronomical Institutes of Czechoslovakia 
\def\caa{Chinese Astron. Astrophys.}%
          % Chinese Astronomy and Astrophysics
\def\cjaa{Chinese J. Astron. Astrophys.}%
          % Chinese Journal of Astronomy and Astrophysics
\def\icarus{Icarus}%
          % Icarus
\def\jcap{J. Cosmology Astropart. Phys.}%
          % Journal of Cosmology and Astroparticle Physics
\def\jrasc{JRASC}%
          % Journal of the RAS of Canada
\def\mnras{MNRAS}%
          % Monthly Notices of the RAS
\def\memras{MmRAS}%
          % Memoirs of the RAS
\def\na{New A}%
          % New Astronomy
\def\nar{New A Rev.}%
          % New Astronomy Review
\def\pasa{PASA}%
          % Publications of the Astron. Soc. of Australia
\def\pra{Phys.~Rev.~A}%
          % Physical Review A: General Physics
\def\prb{Phys.~Rev.~B}%
          % Physical Review B: Solid State
\def\prc{Phys.~Rev.~C}%
          % Physical Review C
\def\prd{Phys.~Rev.~D}%
          % Physical Review D
\def\pre{Phys.~Rev.~E}%
          % Physical Review E
\def\prl{Phys.~Rev.~Lett.}%
          % Physical Review Letters
\def\pasp{PASP}%
          % Publications of the ASP
\def\pasj{PASJ}%
          % Publications of the ASJ
\def\qjras{QJRAS}%
          % Quarterly Journal of the RAS
\def\rmxaa{Rev. Mexicana Astron. Astrofis.}%
          % Revista Mexicana de Astronomia y Astrofisica
\def\skytel{S\&T}%
          % Sky and Telescope
\def\solphys{Sol.~Phys.}%
          % Solar Physics
\def\sovast{Soviet~Ast.}%
          % Soviet Astronomy
\def\ssr{Space~Sci.~Rev.}%
          % Space Science Reviews
\def\zap{ZAp}%
          % Zeitschrift fuer Astrophysik
\def\nat{Nature}%
          % Nature
\def\iaucirc{IAU~Circ.}%
          % IAU Cirulars
\def\aplett{Astrophys.~Lett.}%
          % Astrophysics Letters
\def\apspr{Astrophys.~Space~Phys.~Res.}%
          % Astrophysics Space Physics Research
\def\bain{Bull.~Astron.~Inst.~Netherlands}%
          % Bulletin Astronomical Institute of the Netherlands
\def\fcp{Fund.~Cosmic~Phys.}%
          % Fundamental Cosmic Physics
\def\gca{Geochim.~Cosmochim.~Acta}%
          % Geochimica Cosmochimica Acta
\def\grl{Geophys.~Res.~Lett.}%
          % Geophysics Research Letters
\def\jcp{J.~Chem.~Phys.}%
          % Journal of Chemical Physics
\def\jgr{J.~Geophys.~Res.}%
          % Journal of Geophysics Research
\def\jqsrt{J.~Quant.~Spec.~Radiat.~Transf.}%
          % Journal of Quantitiative Spectroscopy and Radiative Trasfer
\def\memsai{Mem.~Soc.~Astron.~Italiana}%
          % Mem. Societa Astronomica Italiana
\def\nphysa{Nucl.~Phys.~A}%
          % Nuclear Physics A
\def\physrep{Phys.~Rep.}%
          % Physics Reports
\def\physscr{Phys.~Scr}%
          % Physica Scripta
\def\planss{Planet.~Space~Sci.}%
          % Planetary Space Science
\def\procspie{Proc.~SPIE}%
          % Proceedings of the SPIE
\let\astap=\aap
\let\apjlett=\apjl
\let\apjsupp=\apjs
\let\applopt=\ao
\addtolength{\parskip}{2mm}



% \usepackage[font={small,it}, labelfont={small,bf}]{caption}
\definecolor{dark-gray}{gray}{0.15}
\DeclareCaptionFont{gray}{\color{dark-gray}}
% \definecolor{dark-blue}{rgb}{0,0,0.3}
% \DeclareCaptionFont{blue}{\color{dark-blue}}
% \usepackage[labelfont={sc}, format=plain,indention=.5cm]{caption}
\usepackage[font=gray, labelfont=bf, textfont=sl, format=plain, labelsep=endash]{caption}


\usepackage{etoolbox}
\patchcmd{\headrule}{\hrule}{\color{white}\hrule}{}{}

% EGP: removed footnote edit:
% \def\footnotelayout{\color{RUGblue}}

\newcommand{\comment}[1] {{\color{Cerulean}{#1}}}
% EGP: removed footnote edit:
% \renewcommand{\thefootnote}{\textcolor{RUGbluegrey}{\arabic{footnote}}}
%\newcommand{\chapterabstract}[1]{\vskip 1em\centerline{\begin{minipage}{0.8\textwidth}ABSTRACT - #1 \end{minipage}}\vskip 1em}
\newcommand{\chapterabstract}[1]{\textbf{#1}}
\newcommand{\chapterauthors}[1]{\vskip 0.8em\centerline{\begin{minipage}{0.9\textwidth}{ \color{RUGblue}\bf #1 }\end{minipage}}\vskip 1em}
\newcommand{\chapterpublish}[1]{\centerline{\begin{minipage}{0.9\textwidth}{ \color{RUGblue} #1 }\end{minipage}}\vskip 1em}
\newcommand{\chapterstatus}[1]{\textit{#1}}

\newcommand{\abs}[1]{\mid #1 \mid}

\usepackage{multirow}


% \renewcommand*{\bibfont}{\small}  % EGP: weggehaald, compileerde niet op Mac



% New definition of square root:
% it renames \sqrt as \oldsqrt
\let\oldsqrt\sqrt
% it defines the new \sqrt in terms of the old one
\def\sqrt{\mathpalette\DHLhksqrt}
\def\DHLhksqrt#1#2{%
\setbox0=\hbox{$#1\oldsqrt{#2\,}$}\dimen0=\ht0
\advance\dimen0-0.2\ht0
\setbox2=\hbox{\vrule height\ht0 depth -\dimen0}%
{\box0\lower0.4pt\box2}}


%\newcommand{\chapterauthors}[1]{\vskip -1em \centerline{\begin{minipage}{0.8\textwidth}\bf #1 \end{minipage}}\}}


% One can put words in the hyphenation command to indicate where they can or
% cannot be hyphenated. See http://www.giss.nasa.gov/tools/latex/ltx-244.html
% \hyphenation{}


% TM: Added to get better figure placements
% EGP: moved from thesis.tex to model.tex
\renewcommand{\topfraction}{1.00} % max fraction of floats at top
\renewcommand{\bottomfraction}{0.99}  % max fraction of floats at bottom
\renewcommand{\textfraction}{0.00}  % allow minimal text w. figs
\renewcommand{\floatpagefraction}{0.7}  % require fuller float pages


% EGP: todonotes, handy stuff, though tikz requirement makes it slow to compile
\usepackage{todonotes}
\newcommand{\todoi}[1] {\todo[inline]{#1}}
\newcommand{\todog}[1] {\todo[inline, color=green!40]{#1}}
\newcommand{\todor}[1] {\todo[inline, color=red!40]{#1}}
\newcommand{\todolow}[1] {\todo[inline, color=gray!40]{Low priority:\\#1}}
\newcommand{\todohi}[1] {\todog{High priority:\\#1}}
\newcommand{\torien}[1] {\todor{Need help from Rien:\\#1}}
\newcommand{\todofig}[1] {\todo[inline, color=magenta]{Redo figure:\\#1}}


% EGP: used for configurable enumerate counters
\usepackage{enumerate}
\usepackage[inline]{enumitem}

% EGP: for vertical-bar "evaluated at" notation that looks better than with
%      \substack (see http://tex.stackexchange.com/a/122333/66614)
\usepackage{stackengine}

% EGP: for source code
\usepackage{listings}

% EGP: nice 3D coordinate axis diagrams (tikz is dependency)
\usepackage{tikz}
\usepackage{tikz-3dplot}

% EGP: authors and affiliations
\usepackage{authblk}

% EGP: for side of the page colored boxes
% \input{thumbs_48648_66614}
\usepackage{thumbs}

%%%%%%%%%%%%%%%%%%%%%%%%%%%%%%%%%%%%%%%%%%%%%%%%%%%%%%%%%%%%%%%%%%%%%%%%
% Need one style file (fncychap.sty) for chapter cover pages:
%%%%%%%%%%%%%%%%%%%%%%%%%%%%%%%%%%%%%%%%%%%%%%%%%%%%%%%%%%%%%%%%%%%%%%%%
%Options: Sonny, Lenny, Glenn, Conny, Rejne, Bjarne, Bjornstrup

% EGP: load fncychap after any edits to geometry and lengths!
\usepackage[Bjornstrup]{fncychap}



%%%%%%%%%%%%%%%%%%%%%%%%%%%%%%%%%%%%%%%%%%%%%%%%%%%%%%%%%%%%%%%%%%%%%%%
% END
%%%%%%%%%%%%%%%%%%%%%%%%%%%%%%%%%%%%%%%%%%%%%%%%%%%%%%%%%%%%%%%%%%%%%%%%
