%!TEX root = ../thesis.tex
% keep empty line

% keep empty line
\markboth
{Acknowledgments}
{Acknowledgments}
\phantomsection

\selectlanguage{english}
% \selectlanguage{dutch}

\chapter*{Acknowledgments}
\addcontentsline{toc}{chapter}{Acknowledgments}

At the ``end'' of my graduate years (or so I thought, early in 2014), when I was deciding what to do next, I did some thinking on what I might miss most about astronomy if I were to switch careers.
The answer was clear and simple: the people.
For more than a decade now, I've felt truly at home at the Kapteyn Institute and in the astronomical/cosmological community.
For some reason, astronomy attracts a wonderful variation of unique and colorful personalities.
Many of you I've come to call good friends and I hope that we will keep in touch for many years to come.

Not only did you all greatly contribute to a pleasant working environment over the past years, but many of you also contributed to this thesis in some way or another.
I owe all of you my gratitude.
First and foremost among these are, of course, my three promotores.

Rien, you are an inspiring teacher.
Granted, you sometimes lose yourself (and always lose track of time) in your energetic sermons on the gospel of the Large Scale Structure, but more often than not, you take your students along with you.
This is perhaps one of the most defining qualities of your style of teaching: your ability to incite wonderment and fascination.
You are also a great scholar with a broad and deep knowledge of the literature and a boundless curiosity.
My daily research work under your supervision has been a great pleasure and I thank you for all the effort you put in and the many things you taught me.

Paco, your Bayesian outlook on cosmology is, in my opinion, not only very elegant but simply the way to go forward for cosmology.
I am very happy that you were willing to teach me this approach to doing science.
I really enjoyed our conversations during my Potsdam visits and it meant a great deal to me that you entrusted me with many of your ideas.
I think such trust is the key to successful collaboration.
I learned a lot from you about being a good scientist and for that, and your crucial support in putting our framework together, I thank you.

Jelle, I often regret not having finished the observational work we started.
Nonetheless, I am very grateful for the (too) short time I spent under your supervision at SRON.\@
Our successful Suzaku observing proposal was one of proudest moments of my astronomy career.
I want to thank you for your support, especially during the intense weeks before the proposal deadlines.

% READING COMMITTEE
Special thanks go out to the reading committee, Saleem, Alan and Jounghun.
I do not know whether this is the longest thesis produced at the Kapteyn Institute ever, but it probably ranks near the top, so I thank you for your considerable time investment.

% SUPPORT AND COMPUTER GROUP
Astronomers, at least in the Netherlands, are pretty well off.
One always has to be a bit careful when talking to scientists in other fields, or they might get jealous and start demanding the same luxuries we sometimes take for granted.
Apart from our wonderfully cosy building with its ridiculously large, sunny balcony, one of the biggest benefits of working at the Kapteyn is the support from the administrative and computer groups.
Hennie, Gineke, Jackie, Lucia, Christa and Martine, thank you for your indispensable help in dealing with the ever increasing paperwork and other practical issues that make life at the institute better and for keeping alive the great Kapteyn atmosphere.
Martin, Eite, Wim, Hans and Valentin, thank you for making sure we could always rely on the most important tools of our trade (well, except for telescopes, perhaps), our computers.
Martin and Eite, I especially appreciate the many talks we had over the years and also your great flexibility, helpfulness and openness to new ideas.

% GROUP
Rien was amassing quite an army of students around the time I started.
This made our weekly cosmo group meetings a good moment to reflect together about where you were going and what we could do to help each other.
The atmosphere in our group was friendly, fun, stimulating, cooperative and generally a great pleasure.
This great atmosphere was also taken along on the many conferences that we visited together and where we were joined by many more cosmo-friends.
In my memory's chronological order and probably forgetting people: Bernard, Miguel, Pablo, Erwin, Mark, Esra, Jakob, Burcu, Pratyush, Wojtek, Maciek, Niels, Oliver, Johan, Marius, Job, Mathijs, Matthijs, Matti, Mathijs\footnote{The mathematical nature of our group caused a strong clustering of Math-based names.}, Steven, Punya, thanks for good science and good times.

% KAPTEYN
Some very specific thanks go out for various contributions.
My first paranymph Maarten, for helping me squeeze every last drop of performance out of my computer and for teaching me some valuable lessons on unit testing and c++ programming in general and for many coffees and talks.
My second paranymph Johan, for esthetic advise, both on visualization issues and on writing beautiful code.
Omar, for many good Python tips and plotting advise, among which the idea for using TikZ axes and angles in figure~\ref{fig:3daxesangles}.
Metin, for the density-density histogram code used in figure~\ref{fig:ch_barcode_rsd_img_results_23_denvsden_mean_rsd_reg}.
Hans, for the \LaTeX\ title page template.
Stephan, for the thesis template, for Fish Fridays, Nespressos and many talks, e.g.\ about life after astronomy.
Derek, Hugo, Olof, Eva and Jouke, the good old \texttt{\#coma} crew, for many useful and fun conversations on all the things that interest, anger and amuse nerds like us, but also for not banning me after any of my many monologues/rants.
Special thanks also for your advice and support on many issues.
My office mates, Aycin, Oscar, Pece and Leon for good talks and good atmosphere.
To Leon especially, \url{https://youtu.be/0dcbw4IEY5w} ;)
Giacomo, for many coffees.
Jelle dP and Hiroki, for crucial support for the XMM and Suzaku observing proposals.
Also to Yan and the rest of the high-energy SRON team, for the necessary diversion during those weeks.

All the (above mentioned and other) study mates / colleagues I've come to call good friends over the many Kapteyn years, Stephan, Hans, Eva, Jarno, Keimpe, Marlies, Matthijs, Wouter, Jouke, Gerg\H{o}, Tjitske, Ellen, Hugo, Andr\'{e}, Maarten, Johan, Niels, Mathijs, Aleksandar, thanks for many talks, parties, game nights, beers, full-blown borrels, pub quizes and other assorted activities.
I strongly hope that despite many of us scattering far and wide geographically, we will be able to keep in touch for many years to come.

An important problem in science as a whole, and in astronomy specifically \citep{2016arXiv161003159M}, is that proper scientific software development is still considered career suicide if you're going for tenure.
We should make sure we acknowledge the scientific software that was crucial to our research.
It is impossible to be complete, but certainly the work in this thesis would have been a lot harder without at least iPython \citep{ipython}, NumPy \citep{numpy}, SciPy \citep{scipy}, Matplotlib \citep{matplotlib} and Seaborn \citep{seaborn}.
In addition, I gratefully acknowledge the CIT HPC group for computational resources and support on the Millipede and Peregrine clusters.

% eScience
It's hard to imagine how I could have landed a job better suited to my interests than my current job at NLeSC.\@
Not only that, the generous support for finishing my thesis has been both a great motivator and a great relief from the pressure of having to finish while eating into your savings.
It gave me the time to do things right and to do it without too much stress.
For all this, I thank the NLeSC as a whole, but Jisk and Rob especially for deciding to hire me, even though I had little specialist experience for my first projects.

\selectlanguage{dutch}

Vrienden en familie, Gerwin, Leana, Jos, Sanne, Nick, Ineke, Esther, Hendrik, Ettje, Eibert, Suzanne, Age-Sjoerd, Marlies, Michiel, pa, ma, Michel, Samantha, Lilith, Natasja, Kevin, Dani\"{e}l, Tsjepke, Sippie, Hannie, Sander, Fenna, opa \& oma Poelert, Grietje, Gezinus, Marieke, Arjan, Martin, beppe Jikke, Hennie, en alle anderen, bedankt voor de nodige afleiding en jullie interesse en steun, met name ook voor de steun voor Femke en Simon die ik minder tijd heb kunnen geven dan ik had gewild.
Femke, bedankt voor alles en met name je eindeloze geduld en je hulp bij het plannen van de oneindige eindfase.
Eigenlijk had je m'n vierde promotor moeten zijn.

\selectlanguage{english}

Finally, if I forgot to mention you, please file a complaint via your favorite electronic medium.
In case your complaint is considered valid, you will receive a beverage of choice ;)

